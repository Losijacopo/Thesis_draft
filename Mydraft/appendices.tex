\chapter{Pseudocode of relevant algorithms}
\label{chapter:pseudocode}

Some relevant algorithms, which can be useful for a complete understanding of the project, are proposed here in form of pseudocode.
Specifically, it is worth to provide the indicative pipeline of the instructions implemented for the \textit{lightweight} methods, which result to be the best performing in terms of accuracy and computational time.\\
Moreover, significant in relation to the developed work is the derivative calculation procedure used in the \textit{simulated grid} method.
Specifically, the \textbf{consistency check} function added to the evaluation of the local derivatives is following presented, hence it allows to achieve important improvement during the designing of the algorithm.

%\begin{algorithm}
%	\caption{Point cloud densification lightweight method}
%	\begin{algorithmic}[1]
%		
%	\end{algorithmic} 
%\end{algorithm} 
%
\begin{algorithm}
	\caption{Local derivatives calculation (only one case highlighted)}
	\begin{algorithmic}[1]
		\For {$i = 0, 1, 2, \ldots, lookup\_table.rows$}
			\For {$j = 0, 1, 2, \ldots, lookup\_table.cols$}
				\State Run the basic case, i.e. for the internal grid points
				\If {$i>0 \land j>0 \land i<lookup\_table.rows-1 \land j<lookup\_table.cols-1$}
					\State Get lookup\_table value at position $(i, j)$
					\State It reminds to the row of the data matrix corresponding to the searched point
					\State Define point data
					\State Define top neighbor data
					\State Define bottom neighbor data
					\State Define left neighbor data	
					\State Define right neighbor data
					\State Calculate the four local derivatives
					\State Apply $consistency check$ between left and right derivatives	
					\State Apply $consistency check$ between top and bottom derivatives
				\EndIf
			\EndFor
		\EndFor
	\end{algorithmic} 
	\label{alg:local-deriv-cal}
\end{algorithm} 

Regarding the pipeline presented in Algorithm~\ref{alg:local-deriv-cal}, the other cases are omitted in order to not make the comprehension of the algorithm convoluted, being the procedure for the other conditions similar to the scheme outlined.\\
As a matter of fact, the steps applied are the same for all the grid points.
Basically, each point of the initial grid is represented by a \textit{lookup table}, whose value points to the corresponding row of the matrix, which stores the data related to the initial points.\\
Precisely, in the procedure some particular cases have to be mentioned.
These are related to the points that belong to the borders of the grid, and thus of the table too.
In those situations, there are no all the four neighbors for the considered point.
Therefore, the actual derivatives are evaluated with the available data, the others are defined with the values equal to the opposite derivative but changed in sign. 
Let us consider, as an example, the point on the top-left corner of the grid. 
It has only two neighbors, the right and the bottom ones. 
Hence, the right and bottom local derivatives are exactly estimated computing the differences among the complete data. 
The left and top derivatives are, then, set equal to the ones previously evaluated, where the values are changed in sign.\\
Regarding the \textit{consistency check}, it is worth to present the pseudocode related to its implementation, shown in Algorithm\ref{alg:consistency-check}.
As a matter of fact, it represents an important step of the designed algorithm, allowing to reach interesting result when dealing with the simulated grid, as already explained in Section~\ref{subsection:result-consistency-check}.

\begin{algorithm}
	\label{alg:consistency-check}
	\caption{Consistency check procedure between left and right derivatives to a points}
	\begin{algorithmic}[1]
		\If {$abs(der_{left}.Z) > depth\_threshold$}
			\If {$abs(der_{right}.Z) < depth\_threshold$}
				\State $der_{left} = -der_{right}$
			\Else
				\State Estimate new correct value for the $X_{mt}$ datum of the left derivative
				\State Estimate new correct value for the $x_{px}$ datum of the left derivative
				\State Substitute the values to the initial ones
			\EndIf		
		\EndIf
		\If {$abs(der_{right}.Z) > depth\_threshold$}
			\If {$abs(der_{left}.Z) < depth\_threshold$}
				\State $der_{right} = -der_{left}$
			\Else
				\State Estimate new correct value for the $X_{mt}$ datum of the right derivative
				\State Estimate new correct value for the $x_{px}$ datum of the right derivative
				\State Substitute the values to the initial ones
			\EndIf
		\EndIf
	\end{algorithmic} 
\end{algorithm} 

Moreover, pseudocode related to the general pipeline of the \textit{lightweight} algorithm is worth to be reported, considering that this method resulted to be the best performing among the one tested.

\begin{algorithm}
	\caption{Main pipeline of the \textit{lightweight} algorithm}
	\begin{algorithmic}[1]
		\State Import calibration data and images
		\State Import initial raw point cloud data
		\State Set the initial observed values obtained from the raw data
		\State Define the LaDiMo grid
		\State Define the grid squares
		\State Calculate the internal and external derivatives
		\State Image preprocessing
		\State Set the estimations
		\Procedure{Grid Point estimation}{}
			\State Evaluate specific edge case
			\State Calculate matching cost
			\State Get best estimated values accordingly
		\EndProcedure
		\State Post processing over estimations
	\end{algorithmic} 
	\label{alg:lightweight-method}
\end{algorithm} 

%\begin{figure}[b]
%	\begin{center}
%		\AaltoLogoLarge{1}{!}{aaltoRed}
%	\end{center}
%\end{figure}

