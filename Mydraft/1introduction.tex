\chapter{Introduction}
\label{chapter:intro}

%Introduction tells the motivation, scope, goal and the outcome of the
%work. Anyone should be able to understand it. The preferred order of
%writing your master's thesis is about the same as the outline of the
%thesis: you first discover your problem and write about that, then you
%find out what methods you should use and write about that.  Then you
%do your implementation, and document that, and so on.  However, the
%abstract and introduction are often easiest to write last.  This is
%because these really cover the entire thesis, and there is no way you
%could know what to put in your abstract before you have actually done
%your implementation and evaluation. This means that you have to
%rewrite them in the end of your work.

%By the way, rarely anyone write the thesis from the beginning to the
%end just one time, but the writing is more like process, where every
%piece of text is written at least twice. Be also prepared to delete
%your own text. In the first phase, you can hide it into comments that
%are started with \% but during the writing, the many comments should
%be visible for your helpers, the advisor(s) and supervisor.

%Read the information from the university master's thesis
%pages~\cite{ThesisInstructions} before starting the thesis.  You
%should also go through the thesis grading
%instructions~\cite{ThesisGrading} together with your advisor and/or
%supervisor in the beginning of your work.

%The introduction in itself is rarely very long; two to five pages
%often suffice. It usually has two subsections with titles Problem
%statement and Structure of the Thesis, as follows next.


\section{Problem statement}

Dense and accurate disparity maps are the key factor for obtaining correct depth estimations for many computer vision applications such as autonomous driving, 3D reconstruction and robotics.  
In these fields fast calculations over wide images are required due to the necessity of real-time implementation. 
According to the current benchmark database ranks for stereo matching one of the best performing algorithms in term of calculation cost and accuracy is semi-global matching (SGM)\citep{Hirschmuller2008}. 
Basically, this algorithm uses the Mutual Information (MI) as matching cost. It combines concepts of both local and global stereo matching category. 
Specifically, the local-based methods tend to estimate the disparity image trough a comparison of the matching cost from left and right views of the scene. 
In order to recover from low accuracy proper of the previous strategy, global-based methods try to calculate the disparity values by minimizing an energy function. 
In this context, Semi-Global Matching combines strong factors of global and local approaches allowing to obtain a good trade-off between computational cost and accuracy. \\
Considering the whole algorithm, it can be ideally divided in three different main parts. These are the matching cost evaluation, the directional cost calculation and the last phase regards the post-processing. 

\section{Structure of the Thesis}
\label{section:structure} 

You should use transition in your text, meaning that you should help
the reader follow the thesis outline. Here, you tell what will be in
each chapter of your thesis. Often the thesis does not have as many
chapters as is in this template. For example, environment and
implementation can be combined as well as chapters of evaluation and
discussion.  The rest of this thesis is organized as
follows. Chapter~\ref{chapter:background} gives the background, etc.

