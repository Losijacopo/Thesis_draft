\chapter{Conclusions}
\label{chapter:conclusions}

In the previous chapter some considerations regarding the tasks initially assigned to this project and the final results achieved has been proposed.
In relation to that, it is appropriate to summarize here the main acknowledgements and observations concerning the developed Master's thesis.

\section{General and conclusive analysis}
\label{section:general-final-analysis}

As outlined in Chapter~\ref{chapter:discussion}, the predominant goal set at  the initial stages of the project designing was the development of a novel method aimed at generate a dense 3D point cloud starting from a relatively small cluster of sparse points, thus to accurately describe the analysed environment.\\
This would be then be suitable for several applications, at different level of integration, in the computer vision area, such as object detection, robot manipulation, autonomous driving, monitoring car driver macro physical conditions\footnote{as an example an interesting application is tracking the car driver behaviour, thus to enhance the response of the safety system of the car for unexpected situations.}.\\
This can be reasonably evaluated as not a simple objective to accomplish, also taking into account the multiple algorithms developed so far in this researching area.
Therefore, different strategies to tackle the problem have been conceived since the very beginning of the project. 
Moreover, lots of tests over the algorithms produced have been carried out, in order to have clear and visible responses about the followed path.\\
The final results achieved can be, then, measured as a appropriate and suitable realization of the predetermined goals.
Certainly, as mentioned in Chapter~\ref{chapter:discussion}, further improvements will be added to the current work, thus to make it useful for real time applications, while guaranteeing a high level of accuracy of the outcome.
However, especially regarding the \textit{lightweight} algorithm designed, it clearly prove that, first of all, a high level of accuracy can be accomplished by exploiting its main idea, as displayed by the results shown in Chapter~\ref{chapter:evaluation}
Secondly, through proper improvements of the code, aimed at increasing its execution, real time applications can be considered as feasible.\\
Therefore, the accomplishment of the initial task identified should be regarded as produced, taking into account the main researching aspect of the work carried out.
In relation to that, future improvements of the project have already been planned, thus to make it certainly competitive with the current state-of-the-art algorithm and suitable for the market requirements.

