\chapter{Discussion}
\label{chapter:discussion}

The predetermined main objective of this project was to estimate a 3D dense point cloud starting from an initial sparse set of points and a pair of stereo undistorted images.
Moreover, this overall algorithm should have been executed in real time, in order to be following embedded in a stereo device which has to be suitable for autonomous driving, automotive and object detection related purposes.
Therefore, tests for guarantee a potentiality for this type of usability have to be carried out. \\
Overall, this can be evaluated as a challenging work, also considering the different algorithms developed since the early nineties in this field, which is considered as one of the most researched areas in computer vision, especially nowadays.\\
Moreover, since the initial stages of the project, there was the need of studying and understanding the company device and, thus, find the optimal way to embed it in the algorithm pipeline.
Additionally a novel strategy has to be thought, making it suitable for the requirements of the uses cases identified and competitive with the current methods.
Furthermore, since the beginning there was the decision to focus on a standard approach, differently from most of the latest algorithms, which are based on deep learning techniques.  

\section{Overall discussion over the project developed}
\label{section:overall-discussion}


\section{Drawbacks of the designed algorithm}
\label{section:algorithm-drawbacks}


\section{Further improvements and future work}
\label{section:further-improvements}



